% Please use the skeleton file you have received in the 
% invitation-to-submit email, where your data are already
% filled in. Otherwise please make sure you insert your 
% data according to the instructions in PoSauthmanual.pdf
\documentclass{PoS}

\usepackage[printonlyused,smaller]{acronym}

\title{Batch-oriented software appliances}

\ShortTitle{Batch-oriented software appliances}

\author{Riccardo Murri\\
        GC3: Grid Computing Competence Center\\
        University of Zurich \\
        E-mail: \email{riccardo.murri@gmail.com}}

\author{\speaker{Sergio MAFFIOLETTI}\\
        GC3: Grid Computing Competence Center\\
        University of Zurich \\
         E-mail: \email{sergio.maffioletti@gc3.uzh.ch}}


\abstract{This paper presents AppPot, a system for creating Linux
  software appliances. AppPot can be run as a regular batch or grid
  job and executed in user space, and requires no special
  virtualization support in the infrastructure. The main design goal
  of AppPot is to bring the benefits of a virtualization-based IaaS
  cloud to existing batch-oriented computing
  infrastructures. In particular, AppPot addresses the application
  deployment and configuration on large heterogeneous computing
  infrastructures: users are enabled to prepare their own customized
  virtual appliance for providing a safe execution environment for
  their applications. These appliances can then be executed on
  virtually any computing infrastructure being in a private or public
  cloud as well as any batch-controlled computing clusters the user
  may have access to. We give an overview of AppPot and its features,
  the technology that makes it possible, and report on experiences
  running it in production use within the Swiss National Grid
  infrastructure SMSCG.}

\FullConference{EGI Community Forum 2012 / EMI Second Technical Conference\\
          26-30 March, 2012\\
           Munich, Germany}

\begin{document}

\section{Introduction}
Application deployment and configuration on large heterogeneous
systems is a complex infrastructure issue that requires
coordination among various system administrators, end users as well as
operation teams. This is further complicated when it comes to
scientific applications that are, most of the time, not supported on
many Linux distributions.


Virtualized infrastructures and software appliances provide an
effective solution to these problems but do require a specific
infrastructure and a usage model that is markedly different form the
batch-oriented processing that is still the backbone of scientific
computing.

This paper presents a system (nicknamed ``AppPot'') to bring the
benefits of virtualization-based \acs{IaaS} clouds to existing
batch-oriented computing infrastructures. 

AppPot comes in the form of a set of \acs{POSIX} shell 
scripts that are installed into a GNU/Linux system image, and modify
its start-up sequence to allow controlling task execution via the
kernel boot command-line.  Another utility is provided to invoke an
AppPot system image from the shell command line, and request it to
execute a given command. 

\section{Goals and use cases}
\label{sec:usecases}

The following scenarios are meant as an illustration of AppPot's
intended use cases.  Throughout the paper, we shall describe how the
requirements from these use cases translate into design decisions for
AppPot, and how well the goals have been met.

\subsection{Deployment of complex applications}
\label{sec:usecase-deployment}

Some software packages (notably, many scientific codes) require
complex installation procedures. 
\begin{enumerate}
\item\label{R1} 
  The application depends on software that is not readily
  available on the host operating system.
\item\label{R2} 
  The application has a complex or non-standard compilation
  procedure, and the documentation is scarce.
\end{enumerate}
In a grid infrastructure, this poses an additional problem of scale:
all systems administrators must be conversant with the installation
procedures of every software piece, and every application must be
compatible with all the computing systems available in the
infrastructure. 

\subsection{Running self-developed code}
\label{sec:usecase-development}

A large fraction of research groups are developing their own software
applications; oftentimes for computational experiments that are
ephemeral, or limited in scope to a local group or niche community.

Leveraging the \acl{UML} \cite{uml} virtualization system, AppPot
appliances can run on grid and local clusters as regular batch system
jobs, without the need for sysadmin support or root access.
This solves both the aforementioned problems:
\begin{itemize}
\item AppPot software appliances are a way to implement generic
  application deployment on a computational grid, and especially to
  enable users to provide their own software to the computing cluster:
  a complete AppPot appliance consists of three files, that can be
  copied to the execution cluster with any available mechanism,
  including the ``stage in'' facilities provided by most grid and
  cluster batch systems.
\item Users can use an AppPot \ac{VM} on their computer for coding,
  and then run the same \ac{VM} as a Grid jobs or in a Cloud \ac{IaaS}
  infrastructure for larger tests and production runs.
\end{itemize}

\section{Architecture and usage}
\label{sec:architecture}

An AppPot appliance appears to a user as consisting of a few files:
\begin{enumerate}
\item an \emph{AppPot disk image}, which is a complete GNU/Linux system
  installed in a partition image file (in ``raw'' format);
\item an \acs{UML} Linux kernel;
\item a shell script \texttt{apppot-start} used to run a command-line
  program within the AppPot appliance;
\item a few auxiliary programs that enable optional features of
  AppPot (networking, I/O streams redirection).
\end{enumerate}
All these files can, all or in part, be installed system-wide so that many
users can benefit from a shared installation.

\subsection{Using AppPot}
\label{sec:using}

Users receive an AppPot system image, containing a working
installation of a GNU/Linux distribution.  
Users have full access rights to the AppPot system image thus they can modify
it by installing new software, libraries, reference data e.g.,
their own version of a computational code or a reference dataset that
will be used during the computational analysis.

There are three main usage modes of AppPot, detailed below.

\paragraph{Interactive local execution} In this case AppPot is
started on a local machine; the disk image file as well as input data are
directly available. 
\paragraph{Batch job on a cluster resource} In a typical cluster
setup, appliance image file, \ac{UML} kernel and input data are made
available to the batch cluster execution node; the
\texttt{apppot-start} command is invoked by the batch job to run a
command non-interactively within the AppPot appliance.

\paragraph{Grid job} AppPot can also be executed as a grid job on a
distributed infrastructure. In this case, the disk image file,
execution script and reference data need to be transferred to the
destination node before the execution. This is normally achieved by
specifying those input files as part of the grid job description
file. 

\section{Real-world usage}
\label{sec:usage}

Let's see now how the issues illustrated in Section~\ref{sec:usecases}
can be addressed using AppPot.

\subsection{Deployment of complex applications}
\label{sec:usage-deployment}

AppPot allows the execution of the appliance through \ac{UML}; in this
way, the appliance contains the complex application as well as its
full dependencies and, at the same time, it spares the local site
administrators to certify and monitor the application deployment. The
execution within \ac{UML} guarantees that the user cannot gain no more
privileges than the executing user already can.

\subsection{Running development code}
\label{sec:usage-development}

In case application source code needs a frequent update cycle,
re-deploy the software appliance all the time is not a scalable
deployment model. AppPot provides a snapshot/changes mechanism for
this: users can create a ``changes'' file that encodes the differences
of the locally-modified appliance with a ``base'' one. During AppPot
boot, the \texttt{apppot-init} script will re-create the modified
appliance from the base one, by merging in the changes.

\subsection{Dynamical expansion of clusters}
\label{sec:cloudbursting}

A local cluster execution node can be prepared as an AppPot image that
could executed on an accessible external computing infrastructure (the
EGI infrastructure or any flavor of public clouds). The specific
appliance is submitted each time a site seeks to expand its
resources; the AppPot instance could be customized by the site admin
to be a replica of the standard compute node, that connects back to
the home site using a \acs{VPN}. This is similar
to the ``glide-in'' mechanism implemented in the Condor batch
execution system \cite{thain+livny:2005}. 


\section{Conclusions}
\label{sec:conclusions}

AppPot is currently used in production within the Swiss National Grid
Infrastructure \acs{SMSCG}, supporting several use cases like those
presented in this paper.  We are collecting feedback on the
effectiveness of AppPot in large-scale grid computations; we would
like to stress that such effectiveness is not just a function of
system performance, but should also include consideration of how it
makes large-scale computing more accessible (on the users' side) and
manageable (on the systems administrators side).

\appendix

\section{List of acronyms}
\label{sec:acro}
%auto-ignore

%%
%% List of acronyms
%%
%% As long as the `printonlyused` option is active,
%% we can list here as many acronyms we want, but only those
%% referenced in the text will actually be used.
%%

%% mark these acronyms as "used" so they won't be expanded
\acused{API}%
%\acused{GC3}%
\acused{HTTP}%
\acused{SWITCH}%
\acused{URL}%

\begin{acronym}[MMMMMMM]
  \acro{AAI}{Authentication and Authorization Infrastructure}%
  \acro{AC}{Attribute Certificate\acroextra{ (VOMS, X.509)}}%
  \acro{API}{Application Programming Interface}%
  \acro{ARC}{Advanced Resource Connector}
  \acro{CA}{Certification Authority}%
  \acro{CERN}{Centre Européen pour la Recherche Nucléaire\acroextra{
      (European Center for Nuclear Research)}}%
  \acro{COMPCHEM}{Computational Chemistry}
  \acro{CPU}{Central Processing Unit}
  \acro{CSR}{Certificate Signing Request}%
  \acro{DAG}{Directed Acyclic Graph}
  \acro{DN}{Distinguished Name}%
  \acro{EC2}{Amazon's Elastic Compute Cloud}%
  \acro{ECP}{Enhanced Client or Proxy}%
  \acro{EGEE}{Enabling Grids for E-sciencE}%
  \acro{EGI}{European Grid Initiative}
  \acro{FQAN}{Fully-Qualified Attribute Name\acroextra{ (VOMS)}}%
  \acro{GAMESS}{General Atomic and Molecular Electronic Structure System\acroextra{ (a Computational Chemistry application, see \cite{GAMESS:1993,GAMESS:2005})}}%
  \acro{GASuC}{Grid Applications Porting Group}
  \acro{GC3}{Grid Computing Competence Center, University of Zurich}
  \acro{GEMS}{Grid-Enabled Molecular Simulator}
  \acro{GID}{Group Identifier}
  \acro{HPC}{High-Performance Computing}
  \acro{HTC}{High-Thorughput Computing}
  \acro{HTTP}{HyperText Transfer Protocol}%
  \acro{KVM}{Kernel-based Virtual Machine}%
  \acro{IaaS}{Infrastructure as a Service}
  \acro{ID-WSF}{Identity Domain - Web Service Framework\acroextra{ (Shibboleth)}}% FIXME: check!
  \acro{IGI}{Italian Grid Initiative}
  \acro{IGTF}{International Grid Trust Federation}%
  \acro{IdP}{Identity Provider\acroextra{ (Shibboleth)}}%
  \acro{IP}{Internet Protocol}
  \acro{LHC}{Large Hadron Collider}%
  \acro{MPI}{Message Passing Interface}%
  \acro{NGI}{National Grid Infrastructure}
  \acro{OS}{Operating System}\acused{OS}%
  \acro{PaaS}{Platform as a Service}
  \acro{PBS}{Portable Batch System}
  \acro{PES}{Potential Energy Surface}
  \acro{PKI}{Private Key Infrastructure}%
  \acro{POSIX}{Portable Operating System Interface}%
  \acro{SaaS}{Software as a Service}
  \acro{SAML2}{Security Assertion Markup Language version 2}%
  \acro{SAML}{Security Assertion Markup Language}%
  \acro{SLCS}{Short-Lived Credential Service}%
  \acro{SMP}{Symmetric Multi-Processing}%
  \acro{SMSCG}{Swiss Multi-Science Computational Grid}
  \acro{SOA}{Service-Oriented Architecture}
  \acro{SP}{Service Provider\acroextra{ (Shibboleth)}}%
  \acro{SSO}{Single Sign-On}%
  \acro{SWITCH}{Swiss Academic Network Provider}%
  \acro{SWITCHaai}{Swiss Federated AAI}\acused{SWITCHaai}%
  \acro{TORQUE}{Terascale Open-source Resource and QUEue manager}
  \acro{TCP}{Transmission Control Protocol}\acused{TCP}%
  \acro{TERENA}{Trans-European Research and Education Networking
    Association}%
  \acro{UDP}{User Datagram Protocol}\acused{UDP}%
  \acro{UI}{User Interface}
  \acro{UID}{User Identifier}
  \acro{UML}{User-Mode Linux}
  \acro{URL}{Uniform Resource Locator}%
  \acro{VHO}{Virtual Home Organisation}%
  \acro{VM}{Virtual Machine}
  \acro{VM-MAD}{Virtual Machines Management and Advanced
    Deployment\acroextra{ (project ETHZ.7 funded by the SWITCH AAA track)}}%
  \acro{VOMS}{Virtual Organisation Membership Service}%
  \acro{VO}{Virtual Organization}
  \acro{VPN}{Virtual Private Network}
  \acro{WLCG}{Worldwide {LHC} Computing Grid}%
  \acro{WMS}{Workload Management System\acroextra{ (one
      component of the gLite grid middleware system)}}
  \acro{XACML}{eXtensible Access Control Markup Language}%
  \acro{xRSL}{Extended Resource Specification Language}
\end{acronym}

%%% Local Variables: 
%%% mode: latex
%%% TeX-master: "paper"
%%% End: 



\begin{thebibliography}{99}
  \bibitem{thain+livny:2005} Thain, D. and Tannenbaum, T. and Livny,
    M., Distributed computing in practice: The Condor experience,
    Concurrency and Computation: Practice and Experience, volume 17
    2-4, pp 323--356, 2005.
  \bibitem{uml} Dike J., User Mode Linux, Prentice Hall, 2006, isbn 978-0131865051.
\end{thebibliography}

\end{document}



